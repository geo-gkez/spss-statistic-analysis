\clearpage
\section*{Περίληψη}
Η παρούσα εργασία αφορά τη στατιστική ανάλυση και επεξεργασία δεδομένων με τη χρήση του στατιστικού υπολογιστικού πακέτου SPSS.

Οι μεταβλητές που χρησιμοποιήθηκαν για τη στατιστική ανάλυση είναι:
\begin{itemize}
  \item Πρώτο καρδιακό επεισόδιο \lat{(nominal)}
  \item Ηλικία \lat{(scale)}
  \item ΔΑΠ-Διαστολική Αρτηριακή Πίεση \lat{(scale)}
  \item Χοληστερόλη \lat{(scale)}
  \item Αριθμός τσιγάρων \lat{(scale)}
  \item Επιβίωση μετά από 10 χρόνια \lat{(nominal)}
  \item Οικογενειακό Ιστορικό ΚΕ \lat{(nominal)}
\end{itemize}

\underline{Για τις κατηγορικές μεταβλητές (nominal)} ως μέτρο θέσης δόθηκε η επικρατούσα τιμή \lat{(Mode)} , ως σχετικά διαγράμματα τα Pie Chart και Histogram καθώς και οι πίνακες συχνοτήτων.

\underline{Για τις ποσοτικές μεταβλητές (scale)} ως μέτρα θέσης δόθηκαν η επικρατούσα τιμή (Mode), η διάμεσος (Median) και η μέση τιμή (Mean) , ως μέτρα διασποράς δόθηκαν η τυπική απόκλιση (St.dev) και η διακύμανση (Variance) , ως μέτρα μορφής δόθηκαν οι συντελεστές κύρτωσης (Coefficient of Kurtosis) και ασυμμετρίας (Coefficient of Skewness), ως σχετικά διαγράμματα τα Histogram και θηκόγραμμα (Box plot).

